%!TEX root = Intro-NLP-seminar.tex
\section{Cronología de los Modelos Matemáticos}


\begin{frame}
\frametitle{Modelos Matemáticos}

La aplicación de las Matemáticas a la Mecánica Celeste tiene una larga historia.

\begin{itemize}[<+->]
\item El método científico fue esencial para entender "el sistema del
mundo", a finales del siglo XVII. A partir de un modelo
matemático, y siguiendo el método axiomático (Euclides) se
dedujeron las leyes del movimento de los planetas.
(\alert{Newton}, Philosophiae Naturalis Principia Mathematica, 1687)
\item El estudio del movimiento de los planetas planteó problemas tan
fructíferos que supuso el origen de la Teoría del Caos, o Teoría de
los Sistemas Dinámicos, a finales del siglo XIX.
(\alert{Poincaré}, Les métodes nouvelles de la Mécanique Celeste, 1892)
\item Los métodos de la Teoría de Sistemas Dinámicos son útiles hoy
en día para desarrollar complejas misiones espaciales.
\end{itemize}
\end{frame}





\begin{frame}
\frametitle{Modelos Matemáticos}

\textbf{Johannes Kepler (1571-1630)}

\begin{itemize}[<+->]

\item Matemático y astrónomo alemán, descubrió que la Tierra y el resto de planetas se mueven en órbitas elípticas alrededor del Sol, postulando las tres leyes fundamentales del movimiento planetario.

\end{itemize}

\textbf{Isaac Newton (1643 - 1727)}

\begin{itemize}[<+->]

\item Matemático, físico y astrónomo inglés, demostró las tres leyes de Kepler a partir de las leyes de la dinámica y la Ley de la Gravitación Universal.

\end{itemize}

\end{frame}






\begin{frame}
\frametitle{Modelos Matemáticos}

\textbf{U.J. Le Verrier (1811-1877) y J.C. Adams (1819-1892)}

\begin{itemize}[<+->]

\item Independientemente, predijeron la existencia de un octavo planeta que alteraba la órbita de Urano, descubierto en 1791.
\item El planeta Neptuno fue descubierto por el astrónomo J.G. Galle en 1845, en la posición calculada.
\item Le Verrier descubrió en 1855 una discrepancia en la órbita de Mercurio, que no era predicha por la teoría Newtoniana de la gravitación.

\end{itemize}

\end{frame}









\begin{frame}
\frametitle{Modelos Matemáticos}

\textbf{George Fitzgerald (1851-1901)}

\begin{itemize}[<+->]

\item En 1889 Fitzgerald publicó el primer documento conocido sobre un efecto relativista, alegando que el experimento de Michelson-Morley (velocidad de la luz en "éter") podía ser explicado introduciendo una contracción de la longitud en la dirección del movimiento.

\end{itemize}

\textbf{Hendrik Antoon Lorentz (1853-1928)}

\begin{itemize}[<+->]

\item En 1895 Lorentz publicó una versión de primer orden de las transformaciones de Lorentz, para el que los fenómenos eléctricos y ópticos en un sistema en movimiento son independientes de si los términos de orden  se ignoraban. En estas transformaciones introdujo el concepto de hora local. 

\end{itemize}

\end{frame}












\begin{frame}
\frametitle{Modelos Matemáticos}

\textbf{Henri Poincaré (1854-1912)}

\begin{itemize}[<+->]

\item Inicio el estudio de los sistemas dinámicos desde un punto de vista geométrico.

\item Poincaré derivo las ecuaciones \textit{covariantes de gravitación} que predecían correctamente la dirección del perihelio de Mercurio (Punto de la órbita del planeta más próximo al Sol). 

\item En 1905, Henri Poincaré propuso por primera vez ondas gravitacionales (ondes gravifiques) que emanan de un cuerpo y se propagan a la velocidad de la luz.

\item Tras su muerte del francés, David Hilbert publicó un desarrollo de la ecuación covariante gravitatoria, que da paso a la conocida \textit{ecuación de campo de Einstein}.

\end{itemize}

\end{frame}







\begin{frame}
\frametitle{Modelos Matemáticos}

\textbf{Albert Einstein (1879-1955)}

\begin{itemize}[<+->]

\item  En septiembre de 1905, Einstein publicó su artículo "¿Depende la inercia de un cuerpo de su contenido energético?", en este, Einstein deriva las ecuaciones de Lorentz basándose en su Principio de la relatividad y la constancia de la velocidad de la luz, sin asumir la presencia de un éter (ya que, como con la formulación de Poincaré, no puede detectarse en cualquier caso una velocidad uniforme relativa al éter). 
\item El término "relatividad" (sugerido por Max Planck) resalta la idea de la transformación de las leyes de la Física entre observadores en movimiento relativo entre sí.

\end{itemize}

\end{frame}






