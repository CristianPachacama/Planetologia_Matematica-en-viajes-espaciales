%!TEX root = Intro-NLP-seminar.tex
\section{Code Samples for Common Tasks in NLP}

\begin{frame}
\frametitle{What Programming Languages Can I Use for NLP?}
    
\begin{description}
\item[Java] Apache OpenNLP, \textbf<2>{Stanford NLP}, Lucene, GATE, LingPipe\ldots
\item[Python] NLTK (with a nice textbook)
\item[.NET, PHP] \textbf<2>{Stanford NLP}, Lucene\ldots
\end{description}

%(Java and C\#.NET works quite well)

\uncover<2>{Demonstration: Java and PHP, mostly using Stanford's libraries}

\end{frame}


\subsection{Stemming}

\begin{frame}
\frametitle{Libraries for Stemming}
    
\begin{itemize}
\item Many libraries available \\\url{http://tartarus.org/martin/PorterStemmer/}
\item Or implement your own -- nice scope for Individual Project
\item \textcite{porter:1980} is most famous but there are other algorithms too
\end{itemize}

\end{frame}

\begin{frame}[fragile,allowframebreaks]
\frametitle{Stemming Demo using PorterStemmer.class.php}
    
\begin{lstlisting}[language=PHP]
<?php
    require_once('PorterStemmer.class.php');

    $stem = PorterStemmer::Stem("cats");
    echo "$stem<br/>\n";
    $stem = PorterStemmer::Stem("ponies");
    echo "$stem<br/>\n";
    $stem = PorterStemmer::Stem("produce");
    echo "$stem<br/>\n";
    $stem = PorterStemmer::Stem("producer");
    echo "$stem<br/>\n";
    $stem = PorterStemmer::Stem("producing");
    echo "$stem<br/>\n";
?>
\end{lstlisting}
\end{frame}

\begin{frame}[fragile]
\frametitle{Stemmer Output}

\begin{lstlisting}[columns=fullflexible]
cat
poni
produc
produc
produc
\end{lstlisting}

\end{frame}

\subsection{Stanford Parser}


\begin{frame}
\frametitle{Stanford Parser}

\begin{itemize}
\item \parencite{stanford:parser:2003}
\item Stanford Parser can POS-tag, lemmatize \emph{and} parse!
\item Not always the best results, but widely used \Winkey
\end{itemize}    

\end{frame}



\begin{frame}
\frametitle{Installing and Setting Up}

\begin{description}[Java]

\item[Java]
	\begin{itemize}
	\item Download the Java library from \url{http://nlp.stanford.edu/software/lex-parser.shtml}
	\item Unzip and place somewhere on system e.g.~in \directory{C:/}\\[1em]
	\end{itemize}

\item[PHP]
	\begin{itemize}
	\item Download the Java library first
	\item Download the PHP library from \url{https://github.com/agentile/PHP-Stanford-NLP}
	\item Unzip and place in \directory{C:/xampp/htdocs}\\[1em]
	\end{itemize}

\item[.NET]
	\begin{itemize}
	\item Download the Java library first
	\item Follow instructions at \url{http://sergey-tihon.github.io/Stanford.NLP.NET/}
    \item Class names, function calls etc.~exactly same as Java API
	\end{itemize}
\end{description}

\end{frame}

\begin{comment}
\begin{frame}[fragile]
\frametitle{Java Code Sample (Adapted from TaggerDemo.java)}
    
\begin{lstlisting}[language=Java,basicstyle={\ttfamily\scriptsize},gobble=4,
	morekeywords={String, MaxentTagger, List}]
    String model = "models/english-bidirectional-distsim.tagger";
        String text = "This is some text with many sentences and words. " 
                + "We use it to test our programs.";

    // Initialise the POS tagger
    MaxentTagger tagger = new MaxentTagger(model);

    // Pre-processing: sentence-splitting and tokenising
    List<List<HasWord>> sentences = MaxentTagger.tokenizeText(
        new BufferedReader(new StringReader(text)));

    // Loop through list of sentences
    for (List<HasWord> sentence : sentences) {
    	// POS-tag each sentence
        List<TaggedWord> tSentence = tagger.tagSentence(sentence);

        // Loop through each tagged word
        for (TaggedWord tw: tSentence) {
            System.out.print( tw.word() + "/" + tw.tag() + " " );
        }
        System.out.println();
    }
\end{lstlisting}

\end{frame}

\begin{frame}[fragile]
\frametitle{POS-Tagging Output}
    
\begin{lstlisting}[basicstyle=\ttfamily\footnotesize]
This/DT is/VBZ some/DT text/NN with/IN many/JJ sentences/NNS and/CC words/NNS ./. 
We/PRP use/VBP it/PRP to/TO test/VB our/PRP$ programs/NNS ./. 
\end{lstlisting}

Refer to PTB tagset\\
\url{https://www.ling.upenn.edu/courses/Fall_2003/ling001/penn_treebank_pos.html})

\end{frame}

\begin{frame}[fragile]
\frametitle{POS-Tagging Code Sample for PHP}

\begin{lstlisting}[language=PHP,basicstyle=\ttfamily\footnotesize,showtabs=true,tabsize=4]
<?php
require_once('autoload.php');

// Note that the PHP version can only tag single sentence
// (Write your own splitting function using regular expression)
$text = "This is some text with many sentences and words.";

// Copy these files from the Java version to current folder
$pos = new \StanfordNLP\POSTagger(
	'english-bidirectional-distsim.tagger','stanford-postagger.jar');

$results = $pos->tag(explode(' ', $text));

// Loop through the list of tagged words
foreach ($results as $tagged) {
	// each $tagged is an array: word and tag
	echo "$tagged[0]/$tagged[1] ";
}
echo "<br/>";
?>
\end{lstlisting}


\end{frame}
\end{comment}

%\subsection{Lemmatising}

\subsection{POS-Tagging and Lemmatising}

\begin{frame}[fragile,allowframebreaks]
\frametitle{Java Code Sample}

(\small Make sure \texttt{stanford-parser.jar} and \texttt{stanford-parser-\emph{version}-models.jar} are in the library path)


\begin{lstlisting}[language=Java,basicstyle={\ttfamily\footnotesize}, gobble=4,
    morekeywords={String, LexicalizedParser, List,DocumentPreprocessor,Tree,HasWord,Morphology,StringReader},
    emph={apply,getLeaves,value,parent,lemmaStatic}, frame=lines,
    escapechar=|]
    // Initialise the parser using the English model
    String parserModel = "edu/stanford/nlp/models/lexparser/englishPCFG.ser.gz";
    LexicalizedParser lp = LexicalizedParser.loadModel(parserModel);

    // Text to be processed
    String text = "26 interested students came to the seminar. " 
            + "They signed up quickly.";

    // DocumentPreprocessor performs sentence-splitting and tokenising
    for (List<HasWord> sentence : new DocumentPreprocessor(
        new StringReader(text))) {

        // Apply the parser on each sentence
        Tree parse = lp.apply(sentence);

        // Just need POS-tag and lemma?
        for (Tree leaf : parse.getLeaves()) {
            String surfaceForm = leaf.value();
            String pos = leaf.parent(parse).value();
            String lemma = Morphology.lemmaStatic(surfaceForm, pos, true);
            System.out.print(surfaceForm);
            System.out.print("/");
            System.out.print(lemma);
            System.out.print("/");
            System.out.print(pos);
            System.out.print(" ");
        }
        System.out.println();
    }
\end{lstlisting}

\end{frame}

\begin{frame}[fragile]
\frametitle{POS-Tagging and Lemmatising Output}
    
\begin{lstlisting}
23/23/CD interested/interested/JJ students/student/NNS came/come/VBD to/to/TO the/the/DT seminar/seminar/NN ././. 

They/they/PRP signed/sign/VBD up/up/RP quickly/quickly/RB ././. 
\end{lstlisting}

\end{frame}


\begin{frame}[fragile,allowframebreaks]
\frametitle{PHP Code Sample}

\lstset{morecomment=[s][\color{PaleVioletRed4}]{/*}{*/}}
\begin{lstlisting}[language=PHP,basicstyle={\ttfamily\small},
morekeywords={new}]
<?php
require_once('autoload.php');

// Initialise the parser. 
// Put the .jar files somewhere suitable
$parser = new \StanfordNLP\Parser('stanford-parser.jar', 
    'stanford-parser-3.5.0-models.jar');

$text = "26 interested students came to the seminar. " 
        . "They signed up quickly.";

// parse the text
$result = $parser->parseSentence($text);



/* var_dump $result and you'll see it's an array with 
 * 3 outputs: wordsAndTags, penn, typedDependencies */
var_dump($result);

// If only POS tag and lemma are required:
echo "<ul>";
foreach ($result["wordsAndTags"] as $tagged) {
    // each item is an array of the word and POS
    echo "<li>$tagged[0] ($tagged[1])";
}
echo "</ul>";
?>
\end{lstlisting}

\end{frame}


\begin{frame}[fragile]
\frametitle{It doesn't work on my Windows machine!}
    
\lstset{language=PHP,frame=single,basicstyle=\ttfamily\small,gobble=4}

\begin{itemize}
    \item Error: \texttt{Notice: Undefined offset: 1...}
    \item Solution: Modify \texttt{Parser.php}

    \begin{lstlisting}
    $output = explode("\n\n", trim($this->getOutput()));
    \end{lstlisting}

    to

    \begin{lstlisting}
    $output = explode("\r\n\r\n", trim($this->getOutput()));
    \end{lstlisting}

\end{itemize}

\end{frame}


\begin{frame}[fragile]
\frametitle{The PHP version doesn't return lemmas?}
    
Need to modify \texttt{Parser.php} by adding a line:

\begin{lstlisting}[language=PHP,gobble=8,escapechar=|,basicstyle=\ttfamily\small,frame=lines]
        $cmd = $this->getJavaPath()
            . " $options -cp \""
            . $this->getJar()
            . $osSeparator
            . $this->getModelsJar()
            . '" edu.stanford.nlp.parser.lexparser.LexicalizedParser -encoding UTF-8 -outputFormat "'
            . $this->getOutputFormat()
            . "\" "
            |\textbf{\color{PaleVioletRed4}. '-outputFormatOptions "stem" '}|
            . $parser
            . " "
            . $tmpfname;
\end{lstlisting}

\end{frame}

\begin{frame}
\frametitle{POS-Tagging and Lemmatising Output}
    
\begin{itemize}
\item 26 (CD)
\item interested (JJ)
\item student (NNS)
\item come (VBD)
\item to (TO)
\item the (DT)
\item seminar (NN)
\item . (.)
\end{itemize}

\end{frame}

\begin{frame}
\frametitle{PHP version returns only the 1st sentence?}
    
\begin{itemize}
\item The PHP version only captures the output for 1st sentence
\item Possible to modify \texttt{Parser.php} to return output for all sentences
\item (Try yourself or see me if needed)
\end{itemize}

\end{frame}

\subsection{Parsing}

\begin{frame}
\frametitle{Which parsing structure to use?}

\begin{itemize}
\item If you need to use the tree structure of a text -- I'd recommend the \alert{dependency} structure
\item Shorter tree; shows parent-child between word/lemmas in text
\end{itemize}
\end{frame} 


\begin{frame}[fragile,allowframebreaks]
\frametitle{Java Code Sample}
    
\begin{lstlisting}[language=Java,basicstyle=\ttfamily\footnotesize,gobble=8,
    emph={parse,lp,typedDependenciesCCprocessed,lemmaStatic,dep,gov,reln,index,tag,value},
    morekeywords={TreebankLanguagePack,GrammaticalStructureFactory,GrammaticalStructure,
        List,TypedDependency,IndexedWord,String,Morphology},
        escapechar=|]
        // Continue from earlier Java code 
        // Use the parsed tree to get the typed dependencies
        TreebankLanguagePack tlp = lp.treebankLanguagePack();
        GrammaticalStructureFactory gsf = tlp.grammaticalStructureFactory();
        GrammaticalStructure gs = gsf.newGrammaticalStructure(parse);
        List<TypedDependency> tdl = gs.typedDependenciesCCprocessed();

        // Let's just print out each of the parent-child relationship first
        for (TypedDependency td : tdl) {
            // parent = "governer"
            IndexedWord parent = td.gov();
            String parentWord = parent.value();
            String parentPOS = parent.tag();
            String parentLemma = Morphology.lemmaStatic(
                parentWord, parentPOS, true);
            
            // child = "dependent"
            IndexedWord child = td.dep();
            String childWord = child.value();
            String childPOS = child.tag();
            String childLemma = Morphology.lemmaStatic(
                childWord, childPOS, true);
            
            
            System.out.println(
                "[" + parent.index() + "]" + parentLemma + "/" + parentPOS 
                + " <--" + td.reln().getShortName() + "-- "
                + "[" + child.index() + "]" + childLemma + "/" + childPOS);
        }
        System.out.println();
\end{lstlisting}

\end{frame}


\begin{frame}[fragile]
\frametitle{Dependencies as a List}
    
\begin{lstlisting}[basicstyle=\ttfamily\small]
[3]student/NNS <--num-- [1]23/CD
[3]student/NNS <--amod-- [2]interested/JJ
[4]come/VBD <--nsubj-- [3]student/NNS
[0]root/null <--root-- [4]come/VBD
[7]seminar/NN <--det-- [6]the/DT
[4]come/VBD <--prep-- [7]seminar/NN

[2]sign/VBD <--nsubj-- [1]they/PRP
[0]root/null <--root-- [2]sign/VBD
[2]sign/VBD <--prt-- [3]up/RP
[2]sign/VBD <--advmod-- [4]quickly/RB
\end{lstlisting}

\end{frame}


\begin{frame}[fragile]
\frametitle{Recursively Navigating the Dependency Tree}
    
\begin{lstlisting}[language=Java,basicstyle=\ttfamily\footnotesize]
// recursively go through parent-children links, starting from root
int curParent = 0;
processChildren(curParent, tdl);
System.out.println();

private static void processChildren(int parentID, List<TypedDependency> tdl) {
    for (TypedDependency td: tdl) {
        if (td.gov().index() == parentID) {
            IndexedWord childNode = td.dep();
            // do the processing with childNode's values, example:
            // Remember to lemmatise if necessary!!
            System.out.println("Child of node " + parentID + ": ["  
                + childNode.index() + "] " +  childNode.word() + "/" + childNode.tag());
            // then process childNode's children...
            processChildren(childNode.index(), tdl);
        }
    }   
}
\end{lstlisting}

\end{frame}


\begin{frame}[fragile]
\frametitle{Navigating Parent-Child Output}
    
\begin{lstlisting}
Child of node 0: [4] came/VBD
Child of node 4: [3] students/NNS
Child of node 3: [1] 23/CD
Child of node 3: [2] interested/JJ
Child of node 4: [7] seminar/NN
Child of node 7: [6] the/DT

Child of node 0: [2] signed/VBD
Child of node 2: [1] They/PRP
Child of node 2: [3] up/RP
Child of node 2: [4] quickly/RB
\end{lstlisting} 

\end{frame}


\begin{frame}[fragile,allowframebreaks]
\frametitle{PHP Code Sample}
    
\begin{lstlisting}[language=PHP,basicstyle=\ttfamily\footnotesize,
emph={typedDependencies, processChildren}, escapechar={|},
morekeywords={function}]
$curParent = 0;
echo "<ul>";
processChildren($curParent, $result["typedDependencies"]);
echo "</ul>";

function processChildren($curParent, $tdl) {
    foreach ($tdl as $td) {
        $parent = explode("/", $td[0]["feature"]);
        $parentLemma = $parent[0];
        $parentPOS = $parent[1];
        $parentIndex = $td[0]["index"];

        $child = explode("/", $td[1]["feature"]);
        $childLemma = $child[0];
        $childPOS = $child[1];
        $childIndex = $td[1]["index"];
        $reln = $td["type"];

        if ($parentIndex == $curParentID) {
            // do the processing with childNode's values, example:
            echo "<li>Child of node $curParentID: [$childIndex] " 
                . "$childLemma/$childPOS</li>\n";
            // then process childNode's children...
            processChildren($childIndex, $tdl);
        }
    }
}
\end{lstlisting}

\end{frame}


\begin{frame}[fragile]
\frametitle{By default, no POS in typedDependencies?!}
    
Need to modify \texttt{Parser.php} by adding another option:

\begin{lstlisting}[language=PHP,gobble=8,escapechar=|,basicstyle=\ttfamily\small,frame=lines]
        $cmd = $this->getJavaPath()
            . " $options -cp \""
            . $this->getJar()
            . $osSeparator
            . $this->getModelsJar()
            . '" edu.stanford.nlp.parser.lexparser.LexicalizedParser -encoding UTF-8 -outputFormat "'
            . $this->getOutputFormat()
            . "\" "
            |\textbf{\color{PaleVioletRed4}. '-outputFormatOptions "stem,\textcolor{mLightBrown}{includeTags}" '}|
            . $parser
            . " "
            . $tmpfname;
\end{lstlisting}
\end{frame}


\begin{frame}
\frametitle{Output}
    
\begin{itemize}
\item    Child of node 0: [4] come/VBD
\item    Child of node 4: [3] student/NNS
\item    Child of node 3: [1] 26/CD
\item    Child of node 3: [2] interested/JJ
\item    Child of node 4: [7] seminar/NN
\item    Child of node 7: [6] the/DT
\end{itemize}

\end{frame}

\subsection{Speech Synthesis and Recognition}

\begin{frame}
\frametitle{Speech Synthesis and Recognition}
    
\begin{itemize}
\item Microsoft Speech Platform
    \begin{itemize}
        \item English, Japanese, Chinese, French, Spanish\ldots
    \end{itemize}
\item Android -- Google Speech API
    \begin{itemize}
        \item English, Spanish, Japanese, Indonesian, French, Italian, Korean, Hindi\ldots
    \end{itemize}
\item Read the comprehensive API documentations!
\end{itemize}

\end{frame}


\begin{frame}
\frametitle{A Word on Language Codes}
    
\begin{itemize}
    \item ISO-639 Standard
    \item 2-letter and 3-letter codes

\bigskip
\begin{center}
\begin{tabular}{l >{\ttfamily}c >{\ttfamily}c}
\toprule
Language & 2-letter & 3-letter\\
\midrule
English & en & eng\\
Malay & ms & msa, zsm \textsf{(Standard Malay)}\\
Indonesian & id & ind\\
Chinese & zh & zho \\
Cantonese (Yue) & & yue\\
Hokkien (Min Nan) & & nan\\
French & fr & fra\\
\ldots & \ldots & \ldots\\
\bottomrule
\end{tabular}
\end{center}
\bigskip
\end{itemize}
\end{frame}

\begin{frame}
\frametitle{Can also specify Locales}
    
\begin{itemize}
    \item Can add country code to specify locale

\begin{center}
\begin{tabular}{>{\ttfamily}c l}
\toprule
Language code & Language \\
\midrule
en-US & American English\\
en-UK & British English\\
en-AU & Australian English\\
zh-CN & Mainland China Chinese\\
zh-TW & Taiwanese Chinese\\
zh-HK & Hong Kong Chinese\\
\ldots & \ldots\\
\bottomrule
\end{tabular}
\end{center}

\item (Sometimes underscore instead of dash; sometimes given as separate arguments\ldots)

\end{itemize}
\end{frame}

\subsection{Many Others!}

\begin{frame}
\frametitle{Other examples not shown here}

\begin{itemize}
\item Stanford NER (Java, PHP, .NET)
\item If you know Python, do look up NLTK \parencite{nltk:book}
\item GATE: Available as GUI workbench and as embedded API
\item LingPipe: Some interesting libraries for working with corpora
\item \ldots Many, many more!
\end{itemize}
    


\end{frame}